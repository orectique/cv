\documentclass[11pt,a4paper,sans]{moderncv}        
\moderncvstyle{casual}                             
\moderncvcolor{blue}                               
\usepackage[utf8]{inputenc}                        
\usepackage[english]{babel}			   
\usepackage[scale=0.75]{geometry}
\usepackage[useregional]{datetime2}

\name{Sreehari}{Pulickamadhom Sreedhar}

\title{A characterization of an analyst's dramatis persona}                               \address{Canberra}{Australia} 
\phone[mobile]{+61 450 087 293}                      
\email{spulickamadhom@gmail.com}                               
\homepage{https://www.linkedin.com/in/sreehari-sreedhar/}               

\begin{document}
\recipient{Department of Families, Fairness and Housing}{Victoria State Government} % Change to the company name and location.
\date{\today}
\opening{Dear Mr. King,} % Change to Dear Hiring Manager if necessary.
\closing{Yours sincerely,}
\enclosure[Attached]{Resum\'e}          

\makelettertitle

Thank you again for considering my application for the Senior Data Scientist position on the Social Investment Infrastructure team. 

I'm Sreehari, a policy analyst currently working as a modeller at the Australian Local Government Association. I'm also working towards a Master's in international and development economics at the Crawford School of Public Policy and expect to graduate this semester. 

I have a strong foundation in statistics and econometrics, which I have applied in various roles, including my current position where I develop and maintain small-area cameo models to estimate impacts of various policies on populations. At the core of my current work is a robust parallelized small area estimation algorithm that I developed, which has significantly improved the efficiency and speed of our models. Consequently, the savings in time overhead mean that our analyses can now include more variables, complex interactions, and larger datasets. 

Scaling is a critical aspect of my work, and I have successfully implemented solutions that allow my pipelines to handle larger datasets and more complex scenarios without compromising performance. My experience with parallel computing and optimization techniques has been instrumental in achieving these outcomes.

I first worked with Julia in 2022 and have since made it my primary programming language for abstract modelling and quick prototyping of algorithms. On my GitHub \href{https://github.com/orectique}{(@orectique)}, you can find some of my Julia projects, including large-scale agent-based models and evolutionary algorithms. I must note that I tend to use Julia in conjunction with R and Python, depending on the task at hand. The increasing dependency issues amongst Julia packages have made me cautious about using it for production-level code, but I am hopeful that the situation will improve with time. I'm also proficient in low-code tools like Stata, Power BI, and Excel which I tend to use at either end of the data pipeline for inputting, reporting, and visualization.

On the topic of collaboration, I have worked closely with cross-functional teams, including both technical and non-technical stakeholders. Secondary to git, I believe that effective communication is the most important factor to bringing projects to completion. 

I am excited about the opportunity to contribute to the Social Investment Infrastructure team and explore the rich data the team must hold. The role's focus on developing and enhancing microsimulation models using Julia aligns perfectly with my skills and interests. I am particularly drawn to the prospect of working on projects that have a direct impact on government policies and public services.

\makeletterclosing

\end{document}